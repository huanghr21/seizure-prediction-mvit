\documentclass{article}
\usepackage{ctex}
\usepackage{amsmath}
\usepackage{graphicx}
\usepackage{float}
\usepackage{booktabs}

\begin{document}

\section{癫痫发作预测实验复现相关内容}

\subsection{2 脑电数据库及预处理过程}
\subsubsection{2.1 脑电数据库}
本研究仅选取CHB-MIT数据库用于癫痫发作预测验证,SH-SDU数据库因患者发作前期数据不足30分钟未采用。

\paragraph{CHB-MIT数据库详细信息}
CHB-MIT数据库收集于美国波士顿儿童医院,脑电信号采样频率为256 Hz,分辨率为16位,通道数为23-26不等,采用国际标准10-20系统电极放置和命名方法。包含23名患者的24例病例(病例21号与1号为同一位女性患者时隔一年半的记录),每个病例包含多个连续脑电图文件,单个文件记录时长多为1小时,部分长达2小时以上。数据库总计记录约982.93小时脑电数据,包含198次癫痫发作事件,发作起止时间由经验丰富的临床医生标注。

\subsubsection{2.3 癫痫发作预测方法中的预处理}
\paragraph{2.3.1 数据选取}
- 发作前期定义:癫痫发作开始前30分钟;
- 发作后期定义:癫痫发作结束后5分钟;
- 发作间期定义:除发作前期和发作后期外的其余时间。

在CHB-MIT数据库中选择23个共同通道的脑电数据,舍弃不包含这些通道的数据。选取标准:
1. 发作前期数据满足30分钟的癫痫发作事件直接选取;
2. 发作前期数据大于20分钟但不足30分钟的事件,通过数据增强方法扩充至30分钟;
3. 为保证数据均衡性,选取的发作间期数据与发作前期数据时长相等;
4. 第24个病例因缺乏时间信息,不参与癫痫发作预测实验。

CHB-MIT数据库中癫痫预测研究的训练与测试数据集选取信息如表2-4所示:

\begin{table}[H]
\centering
\caption{癫痫预测研究在CHB-MIT数据库中的训练与测试数据集选取信息}
\label{tab:data_selection_prediction}
\begin{tabular}{ccccc}
\toprule
编号 & 选取的发作次数 & 编号 & 选取的发作次数 \\
\midrule
01 & 7 & 13 & 4 \\
02 & 3 & 14 & 6 \\
03 & 6 & 15 & 9 \\
04 & 3 & 16 & 4 \\
05 & 5 & 17 & 3 \\
06 & 10 & 18 & 4 \\
07 & 3 & 19 & 2 \\
08 & 5 & 20 & 6 \\
09 & 4 & 21 & 4 \\
10 & 7 & 22 & 3 \\
11 & 2 & 23 & 6 \\
12 & 7 & 24 & 0 \\
\bottomrule
\end{tabular}
\end{table}

\paragraph{2.3.2 数据增强}
针对CHB-MIT数据库中部分发作前期数据不足30分钟及连续记录文件间存在信号缺失的问题,提出随机分割和随机组合的脑电信号重组方法:
1. 以1秒为单位随机切分现有发作前期脑电数据,切分时间段与缺失时间一致;
2. 将切分后的片段随机打乱并填充到缺失位置,完成数据扩充;
3. 数据分段:采用无重叠的4s滑动窗口对发作前期和间期数据进行分段,每段数据大小为$(C, 1024)$,其中$C$为通道数,1024为采样点数。

\paragraph{2.3.3 时频特征提取(S变换)}
\subparagraph{S变换定义}
对于一维信号$x(t)$,S变换定义为:
\[
S_{x}(\tau, f)=\int_{-\infty}^{+\infty} x(t) w(t-\tau, f) e^{-i 2 \pi f t} d t
\]
其中高斯窗函数:
\[
w(t-\tau, f)=\frac{|f|}{\sqrt{2 \pi}} e^{\frac{-(t-\tau)^{2} f^{2}}{2}}
\]
$\tau$为平移因子,$f$代表频率,$i$为虚数单位。

为获取脑电信号时频域能量分布,计算S变换的平方值:
\[
|S(\tau, f)|^{2}=S(\tau, f) S^{*}(\tau, f)
\]
其中$*$代表共轭运算。

\subparagraph{特征处理步骤}
1. 4秒一段的脑电信号经S变换后得到时频特征矩阵$(T, N)$($T$为时间信息,$N$为频率信息);
2. 因脑电信号有效信息主要分布在50 Hz以下,选取时频矩阵中0-48 Hz部分,得到$(1024, 192)$大小的特征矩阵(1024为时间维度分量,192为频率维度分量);
3. 特征压缩:时间维度32倍求和压缩,频率维度6倍求和压缩,最终得到$(32, 32)$大小的特征矩阵作为模型输入。

\subsection{4 基于S变换和Vision Transformer的癫痫发作预测方法}
\subsubsection{4.1 实验流程}
癫痫发作预测方法总体框架分为预处理、分类模型、后处理三部分:
1. 预处理:原始脑电$\rightarrow$分段$\rightarrow$S变换提取时频特征$\rightarrow$0-48Hz特征选取$\rightarrow$特征压缩;
2. 分类模型:压缩后的特征输入Vision Transformer(ViT)模型$\rightarrow$Softmax输出概率;
3. 后处理:采用k-of-n方法处理输出结果,得到最终预测结果。

实验流程如图4-1所示:
\begin{figure}[H]
\centering
\includegraphics[width=0.8\textwidth]{fig4-1.pdf}
\caption{癫痫发作预测方法整体框架图}
\label{fig:prediction_framework}
\end{figure}

\subsubsection{4.2 Vision Transformer模型构建}
\paragraph{模型核心原理}
ViT基于Transformer架构,核心为自注意力机制,计算公式如下:
\[
Attention (Q, K, V)=Soft max \left(\frac{Q K^{T}}{\sqrt{d_{k}}}\right) V
\]
其中$Q$为查询矩阵,$K$为键值矩阵,$V$为值矩阵,$\sqrt{d_{k}}$为缩放因子。

\paragraph{模型结构细节}
1. 输入处理:多通道脑电频谱特征图平均分割为4个8×8大小的小块,平铺后线性映射为一维向量,与位置向量拼接后作为Transformer Encoder输入;
2. Transformer Encoder:包含LayerNorm层、多头自注意力机制层和多层感知机(Multi-Layer Perceptron);
3. 模型超参数:
   - Transformer Encoder层数:6层;
   - 多头自注意力机制头数:8;
   - 优化器:Adam;
   - 学习率:0.00001;
   - 批处理大小:32;
   - 训练轮数:50;
   - 输出处理:经Softmax层映射为发作前期概率值。

模型结构如图4-2所示:
\begin{figure}[H]
\centering
\includegraphics[width=0.9\textwidth]{fig4-2.pdf}
\caption{用于多通道脑电信号的Vision Transformer模型结构}
\label{fig:vit_model}
\end{figure}

\subsubsection{4.3 后处理过程}
采用k-of-n后处理方法:若$n$个输出结果中包含大于或等于$k$个连续的阳性标签,则将这$n$个输出结果均判定为阳性。本研究中$k=4$,$n=9$。

后处理过程如图4-3所示:
\begin{figure}[H]
\centering
\includegraphics[width=0.8\textwidth]{fig4-3.pdf}
\caption{后处理过程(a)分类网络的输出结果;(b)模型预测标签;(c)经过K-of-N后处理的结果}
\label{fig:post_processing}
\end{figure}

\subsubsection{4.4 实验结果}
\paragraph{4.4.1 系统评估指标}
\subparagraph{基于片段的评估指标}
采用10折交叉验证,评估指标包括灵敏度(Sensitivity)、特异性(Specificity)、准确率(Accuracy)和F1分数:
\[
Sensitivity =\frac{TP}{TP+FN} \times 100\%
\]
\[
Specificity =\frac{TN}{TN+FP} \times 100\%
\]
\[
Accuracy =\frac{TP+TN}{TP+FN+TN+FP} \times 100\%
\]
\[
Recall =\frac{TP}{TP+FN} \times 100\%
\]
\[
Precision =\frac{TP}{TP+FP} \times 100\%
\]
\[
F1\ score =2 \times \frac{P \times R}{P+R}
\]
其中$TP$(True Positive)为正确检测的发作前期片段数,$TN$(True Negative)为正确检测的发作间期片段数,$FP$(False Positive)为发作间期误判为发作前期的次数,$FN$(False Negative)为发作前期误判为发作间期的次数,$P$为Precision,$R$为Recall。

\subparagraph{基于事件的评估指标}
定义癫痫发作预测期(Seizure Prediction Horizon, SPH)和癫痫发作时期(Seizure Occurrence Period, SOP):
- SPH(临床干预期):从预测警报开始到SOP的时间范围,本研究设定为3分钟;
- SOP:癫痫预计发作的时期,本研究设定为30分钟。

评估指标包括:
1. 灵敏度:正确预测的癫痫发作事件占总发作事件的百分比;
2. 预测时间:系统发出警报到癫痫发作开始的时间间隔;
3. 错误预测率(False Prediction Rate, FPR):每小时错误预测癫痫事件的个数。

当模型连续输出超过18个阳性标签时,发出警报。癫痫发作事件的四种预测情况如图4-4所示:
\begin{figure}[H]
\centering
\includegraphics[width=0.8\textwidth]{fig4-4.pdf}
\caption{癫痫发作事件的四种可能预测结果(a)(b)为正确预测;(c)(d)为错误预测}
\label{fig:prediction_cases}
\end{figure}

\paragraph{4.4.2 CHB-MIT数据库实验结果}
\subparagraph{基于片段的实验结果}
表4-1列出了各病例的基于片段评估结果,所有病例灵敏度均超过92%,其中10个病例灵敏度达到100%,平均灵敏度为98.26%;绝大多数病例特异性超过95%,平均特异性为96.89%;5个病例准确率达到100%,平均准确率为97.57%;平均F1分数为0.9761。

\begin{table}[H]
\centering
\caption{CHB-MIT数据库中基于片段的结果}
\label{tab:segment_based_results}
\begin{tabular}{ccccc}
\toprule
编号 & 灵敏度 (\%) & 特异性 (\%) & 准确率 (\%) & F1分数 \\
\midrule
01 & 96.29 & 99.71 & 98.00 & 0.9780 \\
02 & 99.33 & 98.00 & 98.67 & 0.9875 \\
03 & 98.67 & 95.67 & 97.17 & 0.9731 \\
04 & 100.00 & 98.67 & 99.33 & 0.9935 \\
05 & 100.00 & 99.60 & 99.80 & 0.9980 \\
06 & 95.40 & 88.00 & 91.70 & 0.9219 \\
07 & 97.33 & 95.33 & 96.33 & 0.9642 \\
08 & 100.00 & 100.00 & 100.00 & 1.0000 \\
09 & 100.00 & 100.00 & 100.00 & 1.0000 \\
10 & 98.00 & 93.43 & 95.71 & 0.9633 \\
11 & 100.00 & 97.00 & 98.50 & 0.9861 \\
12 & 92.57 & 91.14 & 91.86 & 0.9124 \\
13 & 99.00 & 96.50 & 97.75 & 0.9782 \\
14 & 94.67 & 97.67 & 96.17 & 0.9597 \\
15 & 96.67 & 88.89 & 92.78 & 0.9317 \\
16 & 100.00 & 100.00 & 100.00 & 1.0000 \\
17 & 100.00 & 99.33 & 99.67 & 0.9968 \\
18 & 97.00 & 96.50 & 96.75 & 0.9681 \\
19 & 100.00 & 100.00 & 100.00 & 1.0000 \\
20 & 96.33 & 99.00 & 97.67 & 0.9729 \\
21 & 100.00 & 100.00 & 100.00 & 1.0000 \\
22 & 98.67 & 94.67 & 96.67 & 0.9685 \\
23 & 100.00 & 99.33 & 99.67 & 0.9968 \\
\midrule
平均 & 98.26 & 96.89 & 97.57 & 0.9761 \\
\bottomrule
\end{tabular}
\end{table}

\subparagraph{基于事件的实验结果}
表4-2列出了基于事件的评估结果,21名患者的灵敏度达到100%;76次癫痫发作事件中74次被正确预测,平均灵敏度为97.68%;大多数患者FPR为0.000/h,平均FPR为0.0232/h;最短预测时间为17.69分钟,平均预测时间为24.26分钟。

\begin{table}[H]
\centering
\caption{基于事件的方法在CHB-MIT数据库中的结果}
\label{tab:event_based_results}
\begin{tabular}{ccccc}
\toprule
编号 & 训练的发作事件数量 & 测试的发作事件数量 & 平均预测时间(分钟) & 灵敏度 (\%) & FPR (/h) \\
\midrule
01 & 1 & 6 & 23.73 & 100.00 & 0.0000 \\
02 & 1 & 2 & 21.83 & 100.00 & 0.0000 \\
03 & 3 & 3 & 25.93 & 100.00 & 0.0000 \\
04 & 1 & 2 & 21.23 & 100.00 & 0.0000 \\
05 & 1 & 4 & 25.73 & 100.00 & 0.0000 \\
06 & 5 & 5 & 17.69 & 100.00 & 0.0000 \\
07 & 1 & 2 & 27.53 & 100.00 & 0.0000 \\
08 & 1 & 4 & 25.73 & 100.00 & 0.0000 \\
09 & 1 & 3 & 25.73 & 100.00 & 0.0000 \\
10 & 2 & 5 & 21.41 & 100.00 & 0.0000 \\
11 & 1 & 1 & 25.73 & 100.00 & 0.0000 \\
12 & 2 & 5 & 22.37 & 100.00 & 0.0000 \\
13 & 1 & 3 & 28.73 & 66.67 & 0.3333 \\
14 & 2 & 4 & 20.18 & 100.00 & 0.0000 \\
15 & 4 & 5 & 22.73 & 80.00 & 0.2000 \\
16 & 1 & 3 & 25.73 & 100.00 & 0.0000 \\
17 & 1 & 2 & 25.73 & 100.00 & 0.0000 \\
18 & 2 & 2 & 22.13 & 100.00 & 0.0000 \\
19 & 1 & 1 & 25.73 & 100.00 & 0.0000 \\
20 & 2 & 4 & 25.13 & 100.00 & 0.0000 \\
21 & 1 & 3 & 25.73 & 100.00 & 0.0000 \\
22 & 1 & 2 & 25.73 & 100.00 & 0.0000 \\
23 & 1 & 5 & 25.73 & 100.00 & 0.0000 \\
\midrule
平均 & - & - & 24.26 & 97.68 & 0.0232 \\
\bottomrule
\end{tabular}
\end{table}

\subsubsection{4.5 分析与讨论}
\paragraph{Transformer Encoder层数影响分析}
为探索最佳模型层数,对比了2层、4层、6层和8层Transformer Encoder的ViT模型性能,结果如表4-3所示:

\begin{table}[H]
\centering
\caption{2、4、6和8层的ViT模型在CHB-MIT数据库上的基于片段的平均分类结果}
\label{tab:layer_comparison}
\begin{tabular}{ccccc}
\toprule
层数 & 灵敏度 (\%) & 特异性 (\%) & 准确率 (\%) & F1分数 \\
\midrule
2 & $97.35\pm3.46$ & $95.47\pm5.40$ & $96.41\pm4.07$ & $0.9648\pm0.040$ \\
4 & $97.69\pm3.02$ & $96.44\pm3.61$ & $97.06\pm3.13$ & $0.9709\pm0.031$ \\
6 & $98.26\pm2.07$ & $96.89\pm3.50$ & $97.57\pm2.53$ & $0.9761\pm0.025$ \\
8 & $97.02\pm3.96$ & $94.76\pm5.77$ & $95.89\pm4.59$ & $0.9601\pm0.045$ \\
\bottomrule
\end{tabular}
\end{table}

不同层数模型对23名患者的平均准确率如图4-5所示:
\begin{figure}[H]
\centering
\includegraphics[width=0.9\textwidth]{fig4-5.pdf}
\caption{第2、4、6和8层ViT模型对23名患者的平均准确率}
\label{fig:layer_accuracy}
\end{figure}

分析结论:
1. 模型性能随层数增加先提升,6层时达到最佳;
2. 8层模型因网络过深出现过拟合,性能下降;
3. 仅使用2层网络即可获得良好分类结果,证明方法的稳健性。

\paragraph{与其他方法的性能对比}
表4-4总结了不同方法在CHB-MIT数据库上的性能对比,本文提出的S-transform+ViT方法在基于片段和基于事件的评估指标上均表现更优。

\begin{table}[H]
\centering
\caption{不同方法在CHB-MIT数据集上的性能比较}
\label{tab:method_comparison}
\begin{tabular}{cccccccc}
\toprule
作者 & 方法 & \multicolumn{4}{c}{基于片段的评估} & \multicolumn{2}{c}{基于事件的评估} \\
\cmidrule(lr){3-6} \cmidrule(lr){7-8}
& & 灵敏度 (\%) & 特异性 (\%) & 准确率 (\%) & F1分数 & 灵敏度 (\%) & FPR (/h) \\
\midrule
Alotaiby et al & CSP + LDA & - & - & - & - & 89.00 & 0.3900 \\
Khan et al & CWT + CNN & - & - & - & - & 87.80 & 0.1470 \\
Truong et al & STFT + CNN & - & - & - & - & 81.20 & 0.1600 \\
Yang et al & STFT + RDANet & 89.33 & 93.02 & 92.07 & - & - & - \\
Ryu et al & DenseNet + LSTM & 92.92 & 93.65 & 93.28 & 0.9230 & - & 0.063 \\
Dissanayake et al & GDL & 94.47 & 94.16 & 95.38 & - & - & - \\
Ding et al & CNN + MHA & 86.81 & 86.45 & 86.64 & 0.8680 & 99.45 & 0.0348 \\
Hellar et al & EmDMD + RF & 92.80 & 89.70 & - & 0.9140 & - & - \\
本文 & S-transform + ViT & 98.26 & 96.89 & 97.57 & 0.9761 & 97.68 & 0.0232 \\
\bottomrule
\end{tabular}
\end{table}

\subsubsection{4.6 本章小结}
本研究提出的基于S变换和Vision Transformer的癫痫发作预测方法,通过S变换提取脑电信号时频特征并压缩为(32,32)大小,输入6层Transformer Encoder的ViT模型,经k-of-n后处理得到结果。在CHB-MIT数据库中,基于片段的平均灵敏度98.26%、特异性96.89%、准确率97.57%、F1分数0.9761;基于事件的平均灵敏度97.68%、FPR 0.023/h、预测时间24.26分钟,展现出优越的性能和临床应用潜力。

\subsection{5 总结与展望(相关部分)}
\subparagraph{方法总结}
癫痫发作预测方法结合S变换和ViT的优势:S变换有效提取脑电信号时频特征,ViT通过自注意力机制捕捉全局特征,两者结合提升模型对脑电信号的表示能力,实现高精度预测。

\subparagraph{未来改进方向}
1. 需在更多临床数据库中验证模型通用性和稳定性;
2. 开展跨被试和跨数据库实验,实现模型一次训练适用于所有患者;
3. 将算法部署到FPGA进行硬件加速,满足临床实时检测需求;
4. 优化模型以实现癫痫发作区域定位,减少计算量并提高决策准确性。

\end{document}